%************************************************************************
\section{Background}
\label{sec:background}
% List relevant work by others,
% or preliminary results you have achieved with a detailed and accurate
% explanation and interpretation.
% Include relevant photographs, figures or tables to illustrate the text.
% This section should frame the research questions that your subsequent research will address. 
%************************************************************************
Please add a sentence here that summarizes what this section is about and what the reader can expect.
%************************************************************************
\subsection{Context of the Project and Problem Description} 
\label{sec:context}
%************************************************************************
Explain all the surroundings that are necessary to understand the broader content of your work. If necessary, give a brief introduction to non-HCI research literature as background knowledge. It is best to include a specific scenario\footnote{By scenario, we mean an “informal narrative description”~\cite{carrol1999five}. "It describes human activities or tasks in a story that allows exploration and discussion of contexts, needs, and requirements. It does not necessarily describe the use of software or other technological support used to achieve a goal. Using the vocabulary and phrasing of users means that scenarios can be understood by stakeholders, and they are able to participate fully in development."~\cite{preeceInteractionDesignHumancomputer2015}}. If you are designing a piece of software or interface please specify your target user group and the tasks the users want to perform with your software.

%************************************************************************
\subsection{Related Work}
\label{sec:relatedwork}
%************************************************************************
This section is the result of a simple literature review. This section serves to situate your thesis in the scientific context of other peoples work. Which academic papers exist in your problem area, and how are they related to your work? When placing your thesis in the context of others, you need to consider the dimensions: content and methodology.

\begin{table}[htb]
\small
\colorbox{bamacolor}{
\centering
\begin{tabularx}{\textwidth}{@{} r Y @{}}
	&
	\textbf{Distinction between Bachelor and Master theses}\vspace{2mm}\\
    \textbf{BSc Thesis} &
    A small literature review is mandatory, starting with the papers your supervisor provides. The related work part can also include a mini simiar tool analyzis if you are implementing a specific part of a software. \vspace{2mm}\\
	\textbf{MSc Thesis} &
	A literature review is mandatory. If it is part of your project to choose a specific framework, then you need to conduct a short survey on existing frameworks. This also applies if you want to choose an algorithm to perform a specific task or want to design a study. \vspace{2mm}\\

\end{tabularx}
}
\end{table}

Already in the thesis announcement\footnote{Thesis announcement are descriptions for Open Theses on the HCC website: \url{https://www.mi.fu-berlin.de/en/inf/groups/hcc/theses/open/index.html}}, relevant literature is provided. If this is not the case, please ask your supervisor for articles. Describing the state of the art is essential to identify the gap you would like to fill; thus, the literature's description should clearly result in research gaps. 