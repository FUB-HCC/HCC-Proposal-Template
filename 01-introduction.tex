%************************************************************************
\section{Introduction}
\label{sec:introduction}
% Introduction to the topic.
% Explain why this work is important giving a general introduction to the subject,
% list the basic knowledge needed and outline the purpose of the report. 
%************************************************************************
This section should start with a short thematic introduction of your planned thesis. This includes a description of the context of your thesis, your subject area, and the domain or field in which you aim to do your research. 

In ~\autoref{subsec:motivation} you describe the motivation of your research, why it is relevant, and in ~\autoref{subsec:goal}, you specify the overarching aim of your thesis.

\begin{table}[h]
\small
\colorbox{usethiscolorhere}{
\centering
\begin{tabularx}{\textwidth}{@{} r Y @{}}
	&
	\textbf{Examples}\vspace{2mm}\\
    \textbf{Context} & Machine Learning \vspace{2mm}\\
	\textbf{Subject area} &	Explanations \vspace{2mm}\\
    \textbf{Domain or application area} & Medicine \vspace{2mm}\\
    \textbf{Field of Research} & HCI \vspace{2mm}\\
\end{tabularx}
}
\end{table}

%************************************************************************
\subsection{Motivation}
\label{subsec:motivation}
% Motivation and relevance of the topic.
%************************************************************************
In this section, you should motivate why the problem you are tackling matters and why this is relevant. You want to give the reader enough background knowledge to understand your problem in order to understand your approach. Therefore, introduce the reader to all the necessary fields your research question is part of. Give an overview of why this is interesting. Please illustrate your problem by using an example.

%************************************************************************
\subsection{Goal}
\label{subsec:goal}
% Gap you want to close with your research.
% Goal of your work.
% Your research question.
% List the main research question(s) you want to answer.
% Explain whether your research will provide a definitive answer or simply contribute towards an answer
%************************************************************************
This part of the expose is the most important one since you have to specify the objectives~(goal) of your thesis. The goal of your thesis should be as specific as possible. You also need to outline \emph{what} contribution you are making and \emph{where} you want to make a contribution. By \emph{where} we mean which field of research. Please make sure that the goals are realistic and that you are able to evaluate your goals. Please list the main research question(s) you want to answer. Explain whether your research will provide a
definitive answer or simply contribute towards an answer.

\begin{table}[htb]
\small
\colorbox{bamacolor}{
\centering
\begin{tabularx}{\textwidth}{@{} r Y @{}}
	&
	\textbf{Distinction between Bachelor and Master theses}\vspace{2mm}\\
    \textbf{B. Sc. Thesis} &
    For a B. Sc. thesis, the actual implementation is often a goal. Therefore a thoughtful application of the HCD approach~\cite{gulliksenKeyPrinciplesUsercentred2003, dis20109241} is sufficient. \vspace{2mm}\\
	\textbf{M. Sc. Thesis} &
	In the case of a M. Sc. thesis, a scientific objective needs to be clearly stated and a scientific contribution has to be made. According to \cite{contributionTypes2016} there are seven different contribution types in HCI: theoretical, methodological, empirical, artifact, dataset, survey, and opinion contribution.\vspace{2mm}\\
\end{tabularx}
}
\end{table}

\begin{table}[htb]
\small
\colorbox{usethiscolorhere}{
\centering
\begin{tabularx}{\textwidth}{@{} r Y @{}}
	&
	\textbf{M. Sc. Example taken from ~\cite{Uga1324610}}\vspace{2mm}\\
    \textbf{Goal} & Explore how trustworthy AI can be designed: The goal of this thesis is to suggest design guidelines for the development of trust-inducing explanations to AI-based prediction systems. \vspace{2mm}\\
	\textbf{Contribution} &	Guidelines for designers, that can be applied in a variety of contexts.\vspace{2mm}\\
	\textbf{Contribution Type} & Empirical \vspace{2mm}\\
    \textbf{Research Question} & How can AI predictions be articulated to be trustworthy and actionable?\\
    Sub-questions & --- What explanatory aspects are important to support trust?\\
    & --- What guidance should the articulation provide towards actions to take?\\
    & --- Which presentation formats best support formation of trust?\vspace{2mm}\\
	\textbf{Methodology} & Research through design \vspace{2mm}\\
\end{tabularx}
}
\end{table}

\begin{table}[htb]
\small
\colorbox{usethiscolorhere}{
\centering
\begin{tabularx}{\textwidth}{@{} r Y @{}}
	&
	\textbf{B. Sc. Example taken from~\cite{Joppien2020}}\vspace{2mm}\\
    \textbf{Goal}
    & The goal is to make the diversity of equally valid dimensionality reduction outputs comprehensible through a variety of results so that an individual can make an informed decision. \vspace{2mm}\\
	\textbf{Contribution}
	& -\vspace{2mm}\\
    \textbf{Research Question}
    & Which visual similarity measures help non-technical experts (especially
in the specific scenario described later) to\\
    Sub-questions
    & --- grasp the entire space of possible dimensionality reduction outcomes?\\
    & --- successfully find the most useful visual representation of the information needed for their task?\\
	\textbf{Methodology}
	& HCD\vspace{2mm}\\
\end{tabularx}
}
\end{table}